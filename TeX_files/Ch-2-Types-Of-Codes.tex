\chapter{Types of Codes and Program Execution}

\section{Types of Codes and Program Execution}

\subsection{Source Code}
\textbf{Source Code} is the code that a programmer writes in a high-level programming language
like Python, C, or Java. It is human-readable and easy to understand.

\textbf{Example:}
\begin{lstlisting}[style=pythonstyle, caption=Hello Python Example]
print("Hello, Python!")
\end{lstlisting}

\subsection{Machine Code}
\textbf{Machine Code} is the code that the computer's processor can directly understand.
It is in binary (0s and 1s) and not readable by humans.

\textbf{Example:} 
\begin{verbatim}
10101010 11001100  (just an illustration)
\end{verbatim}

\subsection{Binary Code}
\textbf{Binary Code} is the language of computers consisting of 0s and 1s.
Machine code is a type of binary code. All programs eventually get converted into binary code
so the CPU can execute them.

\subsection{Executable File}
\textbf{Executable File} is a file that contains machine code ready to be run on a computer.
For example, in Windows, programs ending with \texttt{.exe} are executable files.

\subsection{Bytecode}
\textbf{Bytecode} is an intermediate code between source code and machine code.
It is generated when a program is compiled but not yet executed by the computer.
Python converts source code into bytecode first.

\textbf{Example:} \texttt{.pyc} files in Python are Bytecode.

\subsection{Intermediate Code}
\textbf{Intermediate Code} is another name for bytecode.
It is not specific to any computer hardware and can be executed by a virtual machine,
like Python Virtual Machine or Java Virtual Machine.
