\chapter{Variables in Python}
\section{Variables in Python}

A \textbf{variable} is a name used to store data in a program.
In Python, a variable is created when you assign a value to it.

\begin{itemize}
  \item Variables store numbers, text, or other data.
  \item Python does not need a data type while declaring a variable.
  \item The data type is decided automatically.
\end{itemize}

\subsection*{Rules for Naming Variables}

\begin{itemize}
  \item A variable name must start with a \textbf{letter} or \textbf{underscore (\_)}.
  \item It cannot start with a number.
  \item It can contain letters, numbers, and underscores.
  \item Spaces are not allowed.
  \item Python keywords cannot be used as variable names.
  \item Variable names are \textbf{case-sensitive}.
\end{itemize}

\subsection*{Correct Variable Examples}

\begin{lstlisting}[style=pythonstyle,caption={Correct Variable Names}]
age = 25
_name = "Python"
total_marks = 450
price2 = 99.5
is_active = True
\end{lstlisting}

\subsection*{Wrong Variable Examples}

\begin{lstlisting}[style=pythonstyle,caption={Wrong Variable Names}]
2age = 30        # Cannot start with number
total marks = 50 # Space not allowed
class = "A"      # 'class' is a Python keyword
price$ = 100     # Special characters not allowed
\end{lstlisting}

\subsection*{Case Sensitivity Example}

\begin{lstlisting}[style=pythonstyle,caption={Case Sensitivity in Variables}]
value = 10
Value = 20

print(value)  # Output: 10
print(Value)  # Output: 20
\end{lstlisting}