\section{Virtual Machine (VM)}

A \textbf{Virtual Machine (VM)} is a software program that emulates a physical computer.
It provides an environment where programs can run as if they were executed on a real computer.

\begin{itemize}
  \item Acts as a bridge between the program and the hardware.
  \item Makes programs platform-independent.
  \item Examples include: Java Virtual Machine (JVM) and Python Virtual Machine (PVM).
\end{itemize}

\section{Python Virtual Machine (PVM)}

The \textbf{Python Virtual Machine (PVM)} is a component of the Python system
that executes Python bytecode. When you run a Python program:

\begin{enumerate}
  \item Python source code (\texttt{.py}) is compiled into bytecode (\texttt{.pyc}).
  \item The PVM interprets the bytecode and executes it on the computer.
\end{enumerate}

\textbf{Key Points:}
\begin{itemize}
  \item Makes Python platform-independent.
  \item Allows Python programs to run on Windows, Linux, or Mac without modification.
  \item It is a software interpreter, not hardware.
\end{itemize}

\textbf{Note:}  
The Python Virtual Machine (PVM) is a software component that interprets
Python bytecode and executes it on any platform, making Python programs
platform-independent.


\begin{figure}[H] % [H] = exactly here
    \centering
    \includegraphics[width=0.9 \textwidth]{images/diagrams/how-python-program-executed.jpg}
    \caption{Output of Python Program}
    \label{fig:python_output}
\end{figure}
